\section{Recursive Sets \& Relations}

\subsection{Recursive Relations}

A relation $R(x_1, \ldots, x_n)$ is (primitive) recursive if its characteristic 
function

$$f(x_1, \ldots, x_n) = 
    \begin{cases}
        1 & \textrm{if } R(x_1, \ldots, x_n) \textrm{ holds} \\
        0 & \textrm{otherwise}
    \end{cases}
$$

is (primitive) recursive.

\paragraph{Definition by Cases}
Suppose that $f$ is a function defined in the following form:

$$f(x_1, \ldots, x_n) =
    \begin{cases}
        g_1(x_1, \ldots, x_n) & \textrm{if } C_1(x_1, \ldots, x_n) \\
        \vdots & \vdots \\
        g_k(x_1, \ldots, x_n) & \textrm{if } C_k(x_1, \ldots, x_n)
    \end{cases}
$$

$f$ is (primitive) recursive if $C_1, \ldots, C_k$ are mutually exclusive 
(primitive) recursive relations and $g_1, \ldots, g_k$ are (primitive) recursive
(total) functions. 

\begin{proof}
$f$ can be constructed as

$$f(\mathbf{x}) = c_1(\mathbf{x}) \cdot g_1(\mathbf{x}) + \ldots + 
c_k(\mathbf{x}) \cdot g_k(\mathbf{x})$$

where $c_1, \ldots, c_k$ are the characteristic functions of $C_1, \ldots, C_k$ 
and $\mathbf{x}$ denotes the argument list $x_1, \ldots, x_n$. This is 
(primitive) recursive because addition and multiplication are (primitive) 
recursive.
\end{proof}

\paragraph{Substitution}

Given a (primitive) recursive relation $R(y_1, \ldots, y_n)$ and (primitive) 
recursive \textit{total} functions $f_1(x_1, \ldots, x_k), \ldots, f_n(x_1, 
\ldots, x_k)$, the relation $\mathrm{Sub}[R, f_1, \ldots, f_n]$ defined as

$$
\mathrm{Sub}[R, f_1, \ldots, f_n](x_1, \ldots, x_k) 
\leftrightarrow
R(f_1(x_1, \ldots, x_k), \ldots, f_n(x_1, \ldots, x_k))
$$

is (primitive) recursive.

\begin{proof}
The characteristic function $c^*$ of $R^*$ can be constructed by composition

$$c^* = \mathrm{Cn}[c, f_1, \ldots, f_n]$$

where $c$ is the characteristic function of $R$.
\end{proof}

\paragraph{Graph Relations}

The graph relation $F$ of a function $f(x_1, \ldots, x_n)$ is defined as

$$
F(x_1, \ldots, x_n, y)
\leftrightarrow
f(x_1, \ldots, x_n) = y
$$

or in the point-free notation, $F$ can be defined as

$$
\mathrm{Sub}[=, \mathrm{Cn}[f, id^{n+1}_1, \ldots, id^{n+1}_n], id^{n+1}_{n+1}]
$$

which implies that $F$ is (primitive) recursive if $f$ is.

\paragraph{Logical Operations}

There are logical operations that mirror the classical logic connectives. They
define new relations from previous ones. If the relations it operates on are 
(primitive) recursive, then so is the newly defined relation.

Here, $\mathbf{x}$ denotes an argument list $x_1, \ldots, x_n$.

\begin{enumerate}
    \item Given a relation $R(\mathbf{x})$, its negation
    $S(\mathbf{x})$ holds iff $R(\mathbf{x})$ does not.
    
    $$S(\mathbf{x}) \leftrightarrow \neg R(\mathbf{x})$$
    
    \item Given relations $R_1(\mathbf{x})$ and
    $R_2(\mathbf{x})$, their conjunction $S(\mathbf{x})$ holds iff
    $R_1(\mathbf{x})$ and $R_2(\mathbf{x})$ both hold.
    
    $$S(\mathbf{x}) \leftrightarrow R_1(\mathbf{x}) \wedge R_2(\mathbf{x})$$
    
    \item Their disjunction $S(\mathbf{x})$ holds iff
    at least one of $R_1(\mathbf{x})$ and $R_2(\mathbf{x})$ holds.
    
    $$S(\mathbf{x}) \leftrightarrow R_1(\mathbf{x}) \vee R_2(\mathbf{x})$$
    
    \item Given the relation $R(\mathbf{x}, v)$, the bounded universal 
    quantification $S$ is defined as
    
    $$S(\mathbf{x}, u) \leftrightarrow \forall v. (v < u \rightarrow 
    R(\mathbf{x}, v))$$
    
    \item The bounded existential quantification $S$ is defined as
    
    $$S(\mathbf{x}, u) \leftrightarrow \exists v. (v < u  \wedge 
    R(\mathbf{x}, v))$$
\end{enumerate}

\begin{proof} 
Each proof defines the characteristic function of the operator in 
terms of the characteristic function(s) of the given relation(s).

    \begin{enumerate}
        \item $c_{\neg}(\mathbf{x}) = 1 \dotminus c(\mathbf{x})$
        \item $c_{\wedge}(\mathbf{x}) = c_1(\mathbf{x}) \cdot c_2(\mathbf{x})$
        \item $c_{\vee}(\mathbf{x}) = \mathrm{signum}(c_1(\mathbf{x}) + 
        c_2(\mathbf{x}))$
        \item $c_{\forall}(\mathbf{x}, u) = \prod^{u \dotminus 1}_{v = 0} 
        c(\mathbf{x}, 
        v)$
        \item $c_{\exists}(\mathbf{x}, u) = \mathrm{signum}\left( \sum^{u 
        \dotminus 1}_{v = 0} c(\mathbf{x}, v)\right)$
    \end{enumerate}
\end{proof}

\paragraph{Bounded Minimization/Maximization}

Given a (primitive) recursive relation $R$, then the bounded minimization

$$
\mathrm{Min}[R](x_1, \ldots, x_n, w) = 
\begin{cases}
    \begin{aligned}
      &\textrm{the smallest } y \leq w \textrm{ for which } \\
      &R(x_1, \ldots, x_n, y) \textrm{ holds}
    \end{aligned}
    & \textrm{if such a } y \textrm{ exists} \\
    w + 1 & \textrm{otherwise}
\end{cases}
$$

and the bounded maximization

$$
\mathrm{Max}[R](x_1, \ldots, x_n, w) = 
\begin{cases}
    \begin{aligned}
      &\textrm{the largest } y \leq w \textrm{ for which } \\
      &R(x_1, \ldots, x_n, y) \textrm{ holds}
    \end{aligned}
    & \textrm{if such a } y \textrm{ exists} \\
    0 & \textrm{otherwise}
\end{cases}
$$

are (primitive) recursive \textit{total} functions.

\begin{proof}
In the following proofs, use $\mathbf{x}$ as shorthand for $x_1, \ldots, x_n$.

For bounded minimization, consider the relation $\forall t \leq i. \neg R(\mathbf{x}, t)$ and its characteristic function $c(\mathbf{x}, i)$. If a smallest $y \leq w$ exists s.t. $R(\mathbf{x}, y)$, then the first $y$ inputs to the characteristic function will be associated with output $1$ and any inputs from $y$ onwards will be associated with $0$.

\begin{align*}
&c(\mathbf{x}, 0) = \ldots = c(\mathbf{x}, y - 1) = 1 \\
&c(\mathbf{x}, y) = \ldots = c(\mathbf{x}, w) = 0
\end{align*}

However if no such $y$ exists, then all $w + 1$ inputs will be associated with $1$.

$$c(\mathbf{x}, 0) = \ldots = c(\mathbf{x}, w) = 1$$

So, the characteristic function of $\mathrm{Min}[R]$ can be given as the bounded sum over $c(\mathbf{x}, i)$:

$$c_{\mathrm{Min}}(\mathbf{x}, w) = \sum_{i = 0}^{w} c(\mathbf{x}, i)$$

For bounded maximization consider instead the relation $\forall t < i. R(\mathbf{x}, w \dotminus t)$. If a greatest $y \leq w$ exists s.t. $R(\mathbf{x}, y)$, then the last $y$ inputs will be associated to $1$ by its characteristic function, whereas any inputs before will be associated with $0$. If no such $y$ exists then all inputs will be associated with $0$. It is clear then that the characteristic function of $\mathrm{Max}[R]$ can be constructed in the same way as with $\mathrm{Min}[R]$.

\end{proof}
Suppose that $f$ is a recursive function bounded by a primitive recursive function $g$ s.t. the least $y$ with $f(x_1, \ldots, x_n, y) = 0$ is always less than $g(x_1, \ldots, x_n)$. Then using $\mathrm{Min}$, we can state that the function $\mathrm{Mn}[f]$ is not just recursive but actually primitive recursive.

\begin{proof} 
In the following proof, use $\mathbf{x}$ as shorthand for $x_1, \ldots, x_n$.

Let $R(\mathbf{x}, y)$ hold iff $f(\mathbf{x}, y) = 0$. Then we can bound the minimization search using $g$:

$$\mathrm{Mn}[f](\mathbf{x}) = \mathrm{Min}[R](\mathbf{x}, g(\mathbf{x}))$$
\end{proof}