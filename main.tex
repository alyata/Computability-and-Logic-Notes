\documentclass[10pt,twoside,twocolumn]{article}
\usepackage[latin9]{inputenc}

% PAGE FORMATTING %
\usepackage[landscape]{geometry}
\geometry{verbose,tmargin=0.5in,bmargin=0.75in,lmargin=0.5in,rmargin=0.5in}
\setlength{\parskip}{\smallskipamount}
\setlength{\parindent}{0pt}
\setlength{\columnsep}{0.25in}

\usepackage{calc}
\usepackage{amsmath}
\usepackage{amssymb}
\usepackage{esint}
\usepackage{amsthm}
\usepackage{mathrsfs}
\usepackage{amsfonts}
\usepackage{dsfont}
\usepackage{xcolor}

\renewcommand*{\thefootnote}{\fnsymbol{footnote}}

\newcommand{\R}[0]{\mathbb{R}} % real numbers
\newcommand{\Z}[0]{\mathbb{Z}} % integers
\newcommand{\N}[0]{\mathbb{N}} % natural numbers
\newcommand{\nat}[0]{\mathbb{N}} % natural numbers
\newcommand{\Q}[0]{\mathbb{Q}} % rational numbers
\newcommand{\C}[0]{\mathbb{C}} % Complex numbers

% minus with a dot on top
\makeatletter
\newcommand{\dotminus}{\mathbin{\text{\@dotminus}}}

\newcommand{\@dotminus}{%
  \ooalign{\hidewidth\raise1ex\hbox{.}\hidewidth\cr$\m@th-$\cr}%
}
\makeatother

\usepackage{comment}
% uncomment this to temporarily remove proofs
% \excludecomment{proof}

% make it a little easier on the eyes
% \pagecolor[HTML]{2E3440} %black
% \color[HTML]{E5E9F0} %very light grey

% CONTENT %
\begin{document}

\title{Computability \& Logic}
\date{}
\maketitle

\part{Computatibility Theory}

\section{Recursive Functions}

\subsection{Primitive Recursive Functions}

Primitive recursive functions of the form $\nat^k \rightarrow \nat$ are built by
composing primitive functions using function combinators. All primitive 
recursive functions are total functions.

\paragraph{Primitive Functions}

The following primitive functions are primitive recursive.

\begin{itemize}
    \item The zero function
          $$\mathrm{z}(x) = 0$$
    \item The successor function
          $$\mathrm{s}(x) = x'$$
    \item The family of identity functions
          $$\mathrm{id}^n_k(x_1, \ldots. x_n) = x_k$$
          where $n, k \in \nat^+$ and $k \leq n$
\end{itemize}

\paragraph{Function Combinators}

Function combinators construct primitive recursive functions using the given 
primitive recursive functions. There are two:

\begin{itemize}
    \item The Composition combinator ($\mathrm{Cn}$)
          $$\mathrm{Cn}[f, g_1, \ldots, g_k](x_1, \ldots, x_n) = 
          f(g_1(x_1, \dots, x_n), \ldots, g_k(x_1, \ldots, x_n))$$
    \item The Primitive Recursion combinator ($\mathrm{Pr}$)
          \begin{alignat*}{2}
            &\mathrm{Pr}[f, g](x_1, \ldots, x_n, 0) &&= f(x_1, \ldots, x_n) \\
            &\mathrm{Pr}[f, g](x_1, \ldots, x_n, y') &&= g(x_1, \ldots, x_n, y, 
            \mathrm{Pr}[f, g](x_1, \ldots, x_n, y))
          \end{alignat*}

\end{itemize}

\subsection{Recursive Functions}

Recursive functions are composed using the same primitive functions and 
combinators as primitive recursive functions, along with the minimization 
combinator ($\mathrm{Mn}$).

$$
\mathrm{Mn}[f](x_1, \ldots, x_n)=
  \begin{cases}
    y & 
    \begin{aligned}
      \textrm{if } &f(x_1, \ldots, x_n, y) = 0 \textrm{, and for all } t < y \\
      &f(x_1, \ldots, x_n, t) \textrm{ is defined and }\neq 0 
    \end{aligned}\\
    \mathrm{undefined} & \textrm{if there is no such } y
  \end{cases}
$$

The minimization combinator allows for partial functions to be defined. All
recursive (partial) functions are effectively computable. \textbf{Church's 
thesis} implies the converse that all effectively computable functions are 
recursive.


\section{Recursive Sets \& Relations}

\subsection{Recursive Relations}

A relation $R(x_1, \ldots, x_n)$ is (primitive) recursive if its characteristic 
function

$$f(x_1, \ldots, x_n) = 
    \begin{cases}
        1 & \textrm{if } R(x_1, \ldots, x_n) \textrm{ holds} \\
        0 & \textrm{otherwise}
    \end{cases}
$$

is (primitive) recursive.

\paragraph{Definition by Cases}
Suppose that $f$ is a function defined in the following form:

$$f(x_1, \ldots, x_n) =
    \begin{cases}
        g_1(x_1, \ldots, x_n) & \textrm{if } C_1(x_1, \ldots, x_n) \\
        \vdots & \vdots \\
        g_k(x_1, \ldots, x_n) & \textrm{if } C_k(x_1, \ldots, x_n)
    \end{cases}
$$

$f$ is (primitive) recursive if $C_1, \ldots, C_k$ are mutually exclusive 
(primitive) recursive relations and $g_1, \ldots, g_k$ are (primitive) recursive
(total) functions. 

\begin{proof}
$f$ can be constructed as

$$f(\mathbf{x}) = c_1(\mathbf{x}) \cdot g_1(\mathbf{x}) + \ldots + 
c_k(\mathbf{x}) \cdot g_k(\mathbf{x})$$

where $c_1, \ldots, c_k$ are the characteristic functions of $C_1, \ldots, C_k$ 
and $\mathbf{x}$ denotes the argument list $x_1, \ldots, x_n$. This is 
(primitive) recursive because addition and multiplication are (primitive) 
recursive.
\end{proof}

\paragraph{Substitution}

Given a (primitive) recursive relation $R(y_1, \ldots, y_n)$ and (primitive) 
recursive \textit{total} functions $f_1(x_1, \ldots, x_k), \ldots, f_n(x_1, 
\ldots, x_k)$, the relation $\mathrm{Sub}[R, f_1, \ldots, f_n]$ defined as

$$
\mathrm{Sub}[R, f_1, \ldots, f_n](x_1, \ldots, x_k) 
\leftrightarrow
R(f_1(x_1, \ldots, x_k), \ldots, f_n(x_1, \ldots, x_k))
$$

is (primitive) recursive.

\begin{proof}
The characteristic function $c^*$ of $R^*$ can be constructed by composition

$$c^* = \mathrm{Cn}[c, f_1, \ldots, f_n]$$

where $c$ is the characteristic function of $R$.
\end{proof}

\paragraph{Graph Relations}

The graph relation $F$ of a function $f(x_1, \ldots, x_n)$ is defined as

$$
F(x_1, \ldots, x_n, y)
\leftrightarrow
f(x_1, \ldots, x_n) = y
$$

or in the point-free notation, $F$ can be defined as

$$
\mathrm{Sub}[=, \mathrm{Cn}[f, id^{n+1}_1, \ldots, id^{n+1}_n], id^{n+1}_{n+1}]
$$

which implies that $F$ is (primitive) recursive if $f$ is.

\paragraph{Closure Properties}

In addition to substitution, these relational operations define new (primitive)
recursive relations from previously defined ones.

Here, $\mathbf{x}$ denotes an argument list $x_1, \ldots, x_n$.

\begin{enumerate}
    \item Given a relation $R(\mathbf{x})$, its negation
    $S(\mathbf{x})$ holds iff $R(\mathbf{x})$ does not.
    
    $$S(\mathbf{x}) \leftrightarrow \neg R(\mathbf{x})$$
    
    \item Given relations $R_1(\mathbf{x})$ and
    $R_2(\mathbf{x})$, their conjunction $S(\mathbf{x})$ holds iff
    $R_1(\mathbf{x})$ and $R_2(\mathbf{x})$ both hold.
    
    $$S(\mathbf{x}) \leftrightarrow R_1(\mathbf{x}) \wedge R_2(\mathbf{x})$$
    
    \item Their disjunction $S(\mathbf{x})$ holds iff
    at least one of $R_1(\mathbf{x})$ and $R_2(\mathbf{x})$ holds.
    
    $$S(\mathbf{x}) \leftrightarrow R_1(\mathbf{x}) \vee R_2(\mathbf{x})$$
    
    \item Given the relation $R(\mathbf{x}, v)$, the bounded universal 
    quantification $S$ is defined as
    
    $$S(\mathbf{x}, u) \leftrightarrow \forall v. (v < u \rightarrow 
    R(\mathbf{x}, v))$$
    
    \item The bounded existential quantification $S$ is defined as
    
    $$S(\mathbf{x}, u) \leftrightarrow \exists v. (v < u  \wedge 
    R(\mathbf{x}, v))$$
\end{enumerate}

\begin{proof} 
Each proof defines the characteristic function of the operator in 
terms of the characteristic function(s) of the given relation(s). So, if the
given relation(s) are primitive recursive so must the new relation.

    \begin{enumerate}
        \item $c_{\neg}(\mathbf{x}) = 1 \dotminus c(\mathbf{x})$
        \item $c_{\wedge}(\mathbf{x}) = c_1(\mathbf{x}) \cdot c_2(\mathbf{x})$
        \item $c_{\vee}(\mathbf{x}) = \mathrm{signum}(c_1(\mathbf{x}) + 
        c_2(\mathbf{x}))$
        \item $c_{\forall}(\mathbf{x}, u) = \prod^{u \dotminus 1}_{v = 0} 
        c(\mathbf{x}, 
        v)$
        \item $c_{\exists}(\mathbf{x}, u) = \mathrm{signum}\left( \sum^{u 
        \dotminus 1}_{v = 0} c(\mathbf{x}, v)\right)$
    \end{enumerate}
\end{proof}

\paragraph{Bounded Minimization/Maximization}

Given a (primitive) recursive relation $R$, then the bounded minimization

$$
\mathrm{Min}[R](x_1, \ldots, x_n, w) =
\begin{cases}
    \begin{aligned}
      &\textrm{the smallest } y \leq w \textrm{ for which } \\
      &R(x_1, \ldots, x_n, y) \textrm{ holds}
    \end{aligned}
    & \textrm{if such a } y \textrm{ exists} \\
    w + 1 & \textrm{otherwise}
\end{cases}
$$

and the bounded maximization

$$
\mathrm{Max}[R](x_1, \ldots, x_n, w) = 
\begin{cases}
    \begin{aligned}
      &\textrm{the largest } y \leq w \textrm{ for which } \\
      &R(x_1, \ldots, x_n, y) \textrm{ holds}
    \end{aligned}
    & \textrm{if such a } y \textrm{ exists} \\
    0 & \textrm{otherwise}
\end{cases}
$$

are (primitive) recursive \textit{total} functions.

\begin{proof}
In the following proofs, use $\mathbf{x}$ as shorthand for $x_1, \ldots, x_n$.

For bounded minimization, consider the relation $\forall t \leq i. \neg
R(\mathbf{x}, t)$ and its characteristic function $c(\mathbf{x}, i)$. If a
smallest $y \leq w$ exists s.t. $R(\mathbf{x}, y)$, then the first $y$ inputs to
the characteristic function will be associated with output $1$ and any inputs
from $y$ onwards will be associated with $0$.

\begin{align*}
&c(\mathbf{x}, 0) = \ldots = c(\mathbf{x}, y - 1) = 1 \\
&c(\mathbf{x}, y) = \ldots = c(\mathbf{x}, w) = 0
\end{align*}

However if no such $y$ exists, then all $w + 1$ inputs will be associated with
$1$.

$$c(\mathbf{x}, 0) = \ldots = c(\mathbf{x}, w) = 1$$

So, the characteristic function of $\mathrm{Min}[R]$ can be given as the bounded
sum over $c(\mathbf{x}, i)$:

$$c_{\mathrm{Min}}(\mathbf{x}, w) = \sum_{i = 0}^{w} c(\mathbf{x}, i)$$

For bounded maximization consider instead the relation $\forall t < i.
R(\mathbf{x}, w \dotminus t)$. If a greatest $y \leq w$ exists s.t.
$R(\mathbf{x}, y)$, then the last $y$ inputs will be associated to $1$ by its
characteristic function, whereas any inputs before will be associated with $0$.
If no such $y$ exists then all inputs will be associated with $0$. It is clear
then that the characteristic function of $\mathrm{Max}[R]$ can be constructed in
the same way as with $\mathrm{Min}[R]$.

\end{proof}

Suppose that $f$ is a recursive function bounded by a primitive recursive
function $g$ s.t. the least $y$ with $f(x_1, \ldots, x_n, y) = 0$ is always less
than $g(x_1, \ldots, x_n)$. Then using $\mathrm{Min}$, we can state that the
function $\mathrm{Mn}[f]$ is not just recursive but actually primitive
recursive.

\begin{proof} 
In the following proof, use $\mathbf{x}$ as shorthand for $x_1, \ldots, x_n$.

Let $R(\mathbf{x}, y)$ hold iff $f(\mathbf{x}, y) = 0$. Then we can bound the
minimization search using $g$:

$$\mathrm{Mn}[f](\mathbf{x}) = \mathrm{Min}[R](\mathbf{x}, g(\mathbf{x}))$$
\end{proof}

Finally, given a recursive relation $R$, then the total/partial function $r$

$$r(x_1, \ldots, x_n) = \textrm{the least } y \textrm{ such that } R(x_1,
\ldots, x_n, y)$$

is recursive.

\begin{proof}
The function r is just $\mathrm{Mn}[c]$ where $c$ is the characteristic function
of $\neg R$.
\end{proof}

\subsection{Semirecursive Relations}

Given a recursive relation $R$, the relation $S$ defined as

$$S(x_1, \ldots, x_n) \leftrightarrow \exists y. R(x_1, \ldots, x_n, y)$$

is semirecursive. The unbounded quantification implies that a procedure can
compute $S(x_1, \ldots, x_n)$ in finite time if the relation holds, but may
never halt if it does not.

\paragraph{Closure Properties}

As with recursive relations, we can show similar properties of semirecursive
relations.

\begin{enumerate}
    \item Any recursive relation is semirecursive.
    
      \begin{proof}
        Given a recursive relation construct $Q(\mathbf{x}, y) \leftrightarrow
        (R(\mathbf{x}) \land y = y)$. Then we can also define $R$ as
        $R(\mathbf{x}) \leftrightarrow \exists y. Q(\mathbf{x}, y)$. Because $Q$
        is recursive, this implies $R$ is semirecursive.
      \end{proof}

    \item Substituting recursive total functions into a semirecursive relation
    produces a recursive relation.

      \begin{proof}
        Given a semirecursive relation $R(\mathbf{x}) \leftrightarrow \exists y.
        S(\mathbf{x}, y)$ and some functions $f_1, \ldots, f_n$ to substitute,
        construct the relation $Q(\mathbf{x}, y) \leftrightarrow S(f_1(x_1),
        \ldots, f_n(x_n), y)$.  Then $\mathrm{Sub}[R, f_1, \ldots,
        f_n](\mathbf{x}) \leftrightarrow \exists y. Q(\mathbf{x}, y)$ is
        semirecursive because Q is recursive.
      \end{proof}

    \item The conjunction of two semirecursive relations is semirecursive.

      \begin{proof}
        Given semirecursive relations $R_1$ and $R_2$ s.t. $R_i(\mathbf{x})
        \leftrightarrow \exists y.  S_i(\mathbf{x}, y)$, construct the recursive
        relation $Q(\mathbf{x}, w) \leftrightarrow \exists y_1 < w .\exists y_2
        < w.  (S_1(\mathbf{x}, y_1) \land S_2(\mathbf{x}, y_2))$. Then
        $(R_1(\mathbf{x}) \land R_2(\mathbf{x})) \leftrightarrow \exists w.
        Q(\mathbf{x}, w)$ is semirecursive.
      \end{proof}

    \item The disjunction of two semirecursive relations is semirecursive.

      \begin{proof}
        Given semirecursive relations $R_1$ and $R_2$ s.t. $R_i(\mathbf{x})
        \leftrightarrow \exists y.  S_i(\mathbf{x}, y)$, construct the recursive
        relation $Q(\mathbf{x}, y) \leftrightarrow (S_1(\mathbf{x}, y) \lor
        S_2(\mathbf{x}, y))$. Then $(R_1(\mathbf{x}) \lor R_2(\mathbf{x}))
        \leftrightarrow \exists y.  Q(\mathbf{x}, y)$ is semirecursive.
      \end{proof}
    
    \item The relation obtained by bounded universal quantification on a
    semirecursive function is semirecursive.   

      \begin{proof}
        Given a semirecursive relation $R(\mathbf{x}, y) \leftrightarrow \exists
        z. S(\mathbf{x}, y, z)$, construct the recursive relation $Q(\mathbf{x},
        w, v) \leftrightarrow \forall y < w. \exists z < v. S(\mathbf{x}, y,
        z)$. Then $\forall y < w. R(\mathbf{x}, y) \leftrightarrow \exists v.
        Q(\mathbf{x}, w, v)$ is semirecursive.
      \end{proof}

    \item The relation obtained by \textit{unbounded} existential quantification
    on a semirecursive function is semirecursive.

      \begin{proof}
        Given a semirecursive relation $R(\mathbf{x}, y) \leftrightarrow \exists
        z. S(\mathbf{x}, y, z)$, construct the recursive relation $Q(\mathbf{x},
        w) \leftrightarrow \exists y < w. \exists z < w. S(\mathbf{x}, y, z)$.
        Then $\exists y. R(\mathbf{x}, y) \leftrightarrow \exists w.
        Q(\mathbf{x}, w)$ is semirecursive.
      \end{proof}

\end{enumerate}

\paragraph{Complementation Principle}
Also known as Kleene's theorem. If a set and its complement are both
semirecursive, then both are actually recursive.

\begin{proof}
TODO
\end{proof}

\paragraph{First Graph Principle}
If the graph relation of a total or partial function is semirecursive, then the
function itself is recursive.

\begin{proof}
TODO
\end{proof}


\end{document}
